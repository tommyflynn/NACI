\documentclass[]{elsarticle} %review=doublespace preprint=single 5p=2 column
%%% Begin My package additions %%%%%%%%%%%%%%%%%%%
\usepackage[hyphens]{url}

  \journal{International Journal of Nursing Studies} % Sets Journal name


\usepackage{lineno} % add
  \linenumbers % turns line numbering on

\usepackage{graphicx}
%%%%%%%%%%%%%%%% end my additions to header

\usepackage[T1]{fontenc}
\usepackage{lmodern}
\usepackage{amssymb,amsmath}
\usepackage{ifxetex,ifluatex}
\usepackage{fixltx2e} % provides \textsubscript
% use upquote if available, for straight quotes in verbatim environments
\IfFileExists{upquote.sty}{\usepackage{upquote}}{}
\ifnum 0\ifxetex 1\fi\ifluatex 1\fi=0 % if pdftex
  \usepackage[utf8]{inputenc}
\else % if luatex or xelatex
  \usepackage{fontspec}
  \ifxetex
    \usepackage{xltxtra,xunicode}
  \fi
  \defaultfontfeatures{Mapping=tex-text,Scale=MatchLowercase}
  \newcommand{\euro}{€}
\fi
% use microtype if available
\IfFileExists{microtype.sty}{\usepackage{microtype}}{}
\bibliographystyle{elsarticle-harv}
\ifxetex
  \usepackage[setpagesize=false, % page size defined by xetex
              unicode=false, % unicode breaks when used with xetex
              xetex]{hyperref}
\else
  \usepackage[unicode=true]{hyperref}
\fi
\hypersetup{breaklinks=true,
            bookmarks=true,
            pdfauthor={},
            pdftitle={Racialized Network Structures of Clinical Interactions in the ED},
            colorlinks=false,
            urlcolor=blue,
            linkcolor=magenta,
            pdfborder={0 0 0}}
\urlstyle{same}  % don't use monospace font for urls

\setcounter{secnumdepth}{5}
% Pandoc toggle for numbering sections (defaults to be off)


% tightlist command for lists without linebreak
\providecommand{\tightlist}{%
  \setlength{\itemsep}{0pt}\setlength{\parskip}{0pt}}


% Pandoc citation processing
\newlength{\cslhangindent}
\setlength{\cslhangindent}{1.5em}
\newlength{\csllabelwidth}
\setlength{\csllabelwidth}{3em}
\newlength{\cslentryspacingunit} % times entry-spacing
\setlength{\cslentryspacingunit}{\parskip}
% for Pandoc 2.8 to 2.10.1
\newenvironment{cslreferences}%
  {}%
  {\par}
% For Pandoc 2.11+
\newenvironment{CSLReferences}[2] % #1 hanging-ident, #2 entry spacing
 {% don't indent paragraphs
  \setlength{\parindent}{0pt}
  % turn on hanging indent if param 1 is 1
  \ifodd #1
  \let\oldpar\par
  \def\par{\hangindent=\cslhangindent\oldpar}
  \fi
  % set entry spacing
  \setlength{\parskip}{#2\cslentryspacingunit}
 }%
 {}
\usepackage{calc}
\newcommand{\CSLBlock}[1]{#1\hfill\break}
\newcommand{\CSLLeftMargin}[1]{\parbox[t]{\csllabelwidth}{#1}}
\newcommand{\CSLRightInline}[1]{\parbox[t]{\linewidth - \csllabelwidth}{#1}\break}
\newcommand{\CSLIndent}[1]{\hspace{\cslhangindent}#1}




\begin{document}


\begin{frontmatter}

  \title{Racialized Network Structures of Clinical Interactions in the
ED}
    \author[Emory University]{Tommy Flynn\corref{1}}
   \ead{tommy.flynn@emory.edu} 
    \author[Emory University]{Kate Yeager}
   \ead{kyeager@emory.edu} 
    \author[Emory University]{Ymir Vigfusson}
   \ead{ymir@emory.edu} 
    \author[Emory University]{David Wright}
   \ead{david.wright@emory.edu} 
    \author[Emory University]{Dian Evans}
   \ead{dian.evans@emory.edu} 
    \author[Emory University]{Vicki Hertzberg}
   \ead{vhertzberg@emory.edu} 
      \address[Emory University]{1520 Clifton Rd, Atlanta, GA, 30322}
      \cortext[1]{Corresponding Author}
  
  \begin{abstract}
  Network science methodologies offer unique opportunities to study
  real-world human interactions in diverse contexts. The emergency
  department (ED) is a complex dynamic environment in which clinical
  care and patient experience depend on the quality of clinician-patient
  and clinician-clinician encounters, or clinical interactions. Clinical
  interactions (CIs) are multidimensional social phenomena embedded at
  the center of healthcare services and one of the primary pathways
  thought to cause racial healthcare disparities. Evidence of racial
  disparities in the quality of care provided to Black patients compared
  to White patients in U.S. EDs is strong. One roadblock to
  understanding social and clinical mechanisms is the need for research
  methods suited for dynamic complexity inherent to clinical
  interactions in the ED. The purpose of this study is to describe a
  network analysis of clinical interactions (NACI) and the effects of
  patient race on structural network variables in an urban ED in the
  Southern United States. We conducted a secondary analysis of
  observational clinical data to describe how patients' racial
  affiliations relate to CI network structures. Network data were
  generated passively with radiofrequency identification (RFID) tags
  worn by patients and clinicians during their time in the ED. Clinical
  interactions were measured as RFID tag proximity within 1 m. Patient
  demographic and clinical data points were gathered from medical
  records.
  \end{abstract}
  
 \end{frontmatter}

Understanding healthcare disparities is an ongoing research priority for
leading institutions and universities in the United States. Network
science draws on the perspectives, philosophies, and methods of various
qualitative and quantitative research traditions to understand
relationships between objects embedded in complex systems. Focusing a
network science lens on clinical interactions in the complex ED
environment may add to current evidence of patient-clinician dimensions
of racial healthcare disparities. The purpose of this paper is to report
a secondary data analysis of clinical interactions in the Emergency
Department of a large academic hospital in the Southern U.S. using
social network analyses of ED clinical interactions between patients and
healthcare personnel. We report a secondary analysis of longitudinal ED
contact network data. Data were collected previously using a prospective
longitudinal observational study design. The purpose of the parent study
was to describe contact characteristics among patients and staff in the
ED of a busy urban hospital to help inform cross-infection control
measures (Lowery-North et al., 2013). Finding Meaning in Social Network
Structure\\
The meaning of network structures depends on researchers' decisions
about how to define individual actors (i.e., nodes) and their
inter-relational connections (i.e., ties). In fact, the research
question under investigation is considered the guiding force behind
conceptual and operational interpretations of networks. For example,
consider network data generated by a novel contact tracing app designed
to track viral exposures in a given population. This hypothetical
network data would consist of individuals at risk of exposure (nodes),
proximity to other nodes (ties), and exposure status (attribute). As
node nj's number of social contacts (degree centrality) increase, nj's
risk of exposure increases. As more attribute data are available, say
vaccination status, the meanings of network structures (e.g., degree
centrality) are reconsidered. When studying ED network data, we can
assume that a tie between patient and clinician nodes is a part of
service delivery: specifically, a patient-clinician interactions (i.e.,
clinical interactions). In some healthcare settings, clinical
interactions are the building blocks of clinical relationships with
broader implications for quality and equity. Clinical relationships are,
however, beyond the purview of this paper. By defining network
connections as clinician-patient interactions, its centrality is a
measure of--at the least--quantity of direct clinical care, and--we
hypothesize--important dimensions of quality clinical care like
clinician-patient communication, changes in patient acuity, and ED
throughput. Passive Location Collection with Radiofrequency
Identification (RFID) RFID systems are used in a number of hospitals
nationwide (Page, 2007), commonly for supply chain management, passive
patient identification, safe medication administration, patient
tracking, and asset tracking (Yao et al., 2012). RFID systems consist of
small tags or badges that emit radio-frequencies that are picked up by
sensors located strategically so that every RFID badge is always
identifiable to at least 3 sensors for location by triangulation (Yao et
al., 2012). The presence of RFID-enabled healthcare facilities
nationwide will allow for replication of this research, and the
evaluation of equity improvement interventions can be done with some
minor alterations to existing hardware. Compared to other technologies
used for locating resources, the inexpensive and unobtrusive design of
RFID tags make them ideal for healthcare applications. Additionally,
person-to-person proximity data, such as ours, is likely the most
informative sensor-generated data for mapping and studying human
networks (Pentland, 2012). Method This secondary analysis and its parent
study independently received ethics approval by institutional review
board. Sampling strategy and data collection methods were described
previously (Lowery-North et al., 2013) Sampling Sampling strategies in
network research are defined by the purpose of the research at
hand.(Borgatti et al., 2013) Sample size calculations have not yet been
developed for a study designed to compare the networks of various
individuals. In addition, most network analysis research to date does
not use random sampling, instead often describing entire populations
(Borgatti et al., 2013). When studying networks, sampling is more
concerned with defining the bounds of the network and the nature of the
relationships between nodes (Borgatti et al., 2013) The data were
collected with a random sampling of one day shift and one night shift
from a single academic urban ED every week over the course of one year.
A total of 104 shifts of data were collected for the original study
(Lowery-North et al., 2013). The sampling strategy was selected to
minimize sampling bias related to seasonal or weekly fluctuations in
census, acuity, and ED staffing changes (Lowery-North et al., 2013). Due
to various technical complications and personnel issues during the year,
data from a number of shifts scheduled for data collection were deemed
inadequate for analyses. 81 12-hours shifts were available for the
current study.

\hypertarget{references}{%
\section*{References}\label{references}}
\addcontentsline{toc}{section}{References}

\hypertarget{refs}{}
\begin{CSLReferences}{1}{0}
\leavevmode\vadjust pre{\hypertarget{ref-RN150}{}}%
Borgatti, S.P., Everett, M.G., Johnson, J.A., 2013. Analyzing social
networks. Sage Publications Inc, Thousand Oaks, CA.

\leavevmode\vadjust pre{\hypertarget{ref-RN1}{}}%
Lowery-North, D.W., Hertzberg, V.S., Elon, L., Cotsonis, G., Hilton,
S.A., Vaughns 2nd, C.F., Hill, E., Shrestha, A., Jo, A., Adams, N.,
2013. Measuring social contacts in the emergency department. PLoS One 8,
e70854.
doi:\href{https://doi.org/10.1371/journal.pone.0070854}{10.1371/journal.pone.0070854}

\leavevmode\vadjust pre{\hypertarget{ref-RN364}{}}%
Page, L., 2007. Hospitals tune in to RFID. Materials management in
health care. 16, 18--20.

\leavevmode\vadjust pre{\hypertarget{ref-RN259}{}}%
Pentland, A., 2012. The new science of building great teams. Harvard
Business Review 90, 60--69.

\leavevmode\vadjust pre{\hypertarget{ref-RN257}{}}%
Yao, W., Chu, C.-H., Li, Z., 2012. The adoption and implementation of
RFID technologies in healthcare: A literature review. Journal of Medical
Systems 36, 3507--3525.
doi:\href{https://doi.org/10.1007/s10916-011-9789-8}{10.1007/s10916-011-9789-8}

\end{CSLReferences}


\end{document}
