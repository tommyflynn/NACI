% Options for packages loaded elsewhere
\PassOptionsToPackage{unicode}{hyperref}
\PassOptionsToPackage{hyphens}{url}
%
\documentclass[
]{article}
\usepackage{amsmath,amssymb}
\usepackage{lmodern}
\usepackage{iftex}
\ifPDFTeX
  \usepackage[T1]{fontenc}
  \usepackage[utf8]{inputenc}
  \usepackage{textcomp} % provide euro and other symbols
\else % if luatex or xetex
  \usepackage{unicode-math}
  \defaultfontfeatures{Scale=MatchLowercase}
  \defaultfontfeatures[\rmfamily]{Ligatures=TeX,Scale=1}
\fi
% Use upquote if available, for straight quotes in verbatim environments
\IfFileExists{upquote.sty}{\usepackage{upquote}}{}
\IfFileExists{microtype.sty}{% use microtype if available
  \usepackage[]{microtype}
  \UseMicrotypeSet[protrusion]{basicmath} % disable protrusion for tt fonts
}{}
\makeatletter
\@ifundefined{KOMAClassName}{% if non-KOMA class
  \IfFileExists{parskip.sty}{%
    \usepackage{parskip}
  }{% else
    \setlength{\parindent}{0pt}
    \setlength{\parskip}{6pt plus 2pt minus 1pt}}
}{% if KOMA class
  \KOMAoptions{parskip=half}}
\makeatother
\usepackage{xcolor}
\IfFileExists{xurl.sty}{\usepackage{xurl}}{} % add URL line breaks if available
\IfFileExists{bookmark.sty}{\usepackage{bookmark}}{\usepackage{hyperref}}
\hypersetup{
  pdftitle={Network Analysis of Clinical Interactions: ED Network Structures and Patient Race},
  hidelinks,
  pdfcreator={LaTeX via pandoc}}
\urlstyle{same} % disable monospaced font for URLs
\usepackage[margin=1in]{geometry}
\usepackage{color}
\usepackage{fancyvrb}
\newcommand{\VerbBar}{|}
\newcommand{\VERB}{\Verb[commandchars=\\\{\}]}
\DefineVerbatimEnvironment{Highlighting}{Verbatim}{commandchars=\\\{\}}
% Add ',fontsize=\small' for more characters per line
\usepackage{framed}
\definecolor{shadecolor}{RGB}{248,248,248}
\newenvironment{Shaded}{\begin{snugshade}}{\end{snugshade}}
\newcommand{\AlertTok}[1]{\textcolor[rgb]{0.94,0.16,0.16}{#1}}
\newcommand{\AnnotationTok}[1]{\textcolor[rgb]{0.56,0.35,0.01}{\textbf{\textit{#1}}}}
\newcommand{\AttributeTok}[1]{\textcolor[rgb]{0.77,0.63,0.00}{#1}}
\newcommand{\BaseNTok}[1]{\textcolor[rgb]{0.00,0.00,0.81}{#1}}
\newcommand{\BuiltInTok}[1]{#1}
\newcommand{\CharTok}[1]{\textcolor[rgb]{0.31,0.60,0.02}{#1}}
\newcommand{\CommentTok}[1]{\textcolor[rgb]{0.56,0.35,0.01}{\textit{#1}}}
\newcommand{\CommentVarTok}[1]{\textcolor[rgb]{0.56,0.35,0.01}{\textbf{\textit{#1}}}}
\newcommand{\ConstantTok}[1]{\textcolor[rgb]{0.00,0.00,0.00}{#1}}
\newcommand{\ControlFlowTok}[1]{\textcolor[rgb]{0.13,0.29,0.53}{\textbf{#1}}}
\newcommand{\DataTypeTok}[1]{\textcolor[rgb]{0.13,0.29,0.53}{#1}}
\newcommand{\DecValTok}[1]{\textcolor[rgb]{0.00,0.00,0.81}{#1}}
\newcommand{\DocumentationTok}[1]{\textcolor[rgb]{0.56,0.35,0.01}{\textbf{\textit{#1}}}}
\newcommand{\ErrorTok}[1]{\textcolor[rgb]{0.64,0.00,0.00}{\textbf{#1}}}
\newcommand{\ExtensionTok}[1]{#1}
\newcommand{\FloatTok}[1]{\textcolor[rgb]{0.00,0.00,0.81}{#1}}
\newcommand{\FunctionTok}[1]{\textcolor[rgb]{0.00,0.00,0.00}{#1}}
\newcommand{\ImportTok}[1]{#1}
\newcommand{\InformationTok}[1]{\textcolor[rgb]{0.56,0.35,0.01}{\textbf{\textit{#1}}}}
\newcommand{\KeywordTok}[1]{\textcolor[rgb]{0.13,0.29,0.53}{\textbf{#1}}}
\newcommand{\NormalTok}[1]{#1}
\newcommand{\OperatorTok}[1]{\textcolor[rgb]{0.81,0.36,0.00}{\textbf{#1}}}
\newcommand{\OtherTok}[1]{\textcolor[rgb]{0.56,0.35,0.01}{#1}}
\newcommand{\PreprocessorTok}[1]{\textcolor[rgb]{0.56,0.35,0.01}{\textit{#1}}}
\newcommand{\RegionMarkerTok}[1]{#1}
\newcommand{\SpecialCharTok}[1]{\textcolor[rgb]{0.00,0.00,0.00}{#1}}
\newcommand{\SpecialStringTok}[1]{\textcolor[rgb]{0.31,0.60,0.02}{#1}}
\newcommand{\StringTok}[1]{\textcolor[rgb]{0.31,0.60,0.02}{#1}}
\newcommand{\VariableTok}[1]{\textcolor[rgb]{0.00,0.00,0.00}{#1}}
\newcommand{\VerbatimStringTok}[1]{\textcolor[rgb]{0.31,0.60,0.02}{#1}}
\newcommand{\WarningTok}[1]{\textcolor[rgb]{0.56,0.35,0.01}{\textbf{\textit{#1}}}}
\usepackage{graphicx}
\makeatletter
\def\maxwidth{\ifdim\Gin@nat@width>\linewidth\linewidth\else\Gin@nat@width\fi}
\def\maxheight{\ifdim\Gin@nat@height>\textheight\textheight\else\Gin@nat@height\fi}
\makeatother
% Scale images if necessary, so that they will not overflow the page
% margins by default, and it is still possible to overwrite the defaults
% using explicit options in \includegraphics[width, height, ...]{}
\setkeys{Gin}{width=\maxwidth,height=\maxheight,keepaspectratio}
% Set default figure placement to htbp
\makeatletter
\def\fps@figure{htbp}
\makeatother
\setlength{\emergencystretch}{3em} % prevent overfull lines
\providecommand{\tightlist}{%
  \setlength{\itemsep}{0pt}\setlength{\parskip}{0pt}}
\setcounter{secnumdepth}{-\maxdimen} % remove section numbering
\newlength{\cslhangindent}
\setlength{\cslhangindent}{1.5em}
\newlength{\csllabelwidth}
\setlength{\csllabelwidth}{3em}
\newlength{\cslentryspacingunit} % times entry-spacing
\setlength{\cslentryspacingunit}{\parskip}
\newenvironment{CSLReferences}[2] % #1 hanging-ident, #2 entry spacing
 {% don't indent paragraphs
  \setlength{\parindent}{0pt}
  % turn on hanging indent if param 1 is 1
  \ifodd #1
  \let\oldpar\par
  \def\par{\hangindent=\cslhangindent\oldpar}
  \fi
  % set entry spacing
  \setlength{\parskip}{#2\cslentryspacingunit}
 }%
 {}
\usepackage{calc}
\newcommand{\CSLBlock}[1]{#1\hfill\break}
\newcommand{\CSLLeftMargin}[1]{\parbox[t]{\csllabelwidth}{#1}}
\newcommand{\CSLRightInline}[1]{\parbox[t]{\linewidth - \csllabelwidth}{#1}\break}
\newcommand{\CSLIndent}[1]{\hspace{\cslhangindent}#1}
\usepackage{booktabs}
\usepackage{longtable}
\usepackage{array}
\usepackage{multirow}
\usepackage{wrapfig}
\usepackage{float}
\usepackage{colortbl}
\usepackage{pdflscape}
\usepackage{tabu}
\usepackage{threeparttable}
\usepackage{threeparttablex}
\usepackage[normalem]{ulem}
\usepackage{makecell}
\usepackage{xcolor}
\ifLuaTeX
  \usepackage{selnolig}  % disable illegal ligatures
\fi

\title{Network Analysis of Clinical Interactions: ED Network Structures
and Patient Race}
\author{true \and true \and true \and true \and true \and true \and true}
\date{2022-05-09}

\begin{document}
\maketitle

\hypertarget{code}{%
\section{Code}\label{code}}

\textbf{Population from all shifts during the observation year
(2009-2010):}

\begin{Shaded}
\begin{Highlighting}[]
\CommentTok{\# load the population data from the full year, all 730 shifts"}
\NormalTok{population\_730 }\OtherTok{\textless{}{-}} \FunctionTok{read\_rds}\NormalTok{(}\FunctionTok{paste0}\NormalTok{(rdata\_dir, }\StringTok{"/population.Rdata"}\NormalTok{))}
\CommentTok{\# clean race variable for analysis}
\NormalTok{population\_730 }\OtherTok{\textless{}{-}}\NormalTok{ population\_730 }\SpecialCharTok{\%\textgreater{}\%}
  \FunctionTok{mutate}\NormalTok{(}\AttributeTok{race =} \FunctionTok{factor}\NormalTok{(race,}
              \AttributeTok{levels =} \FunctionTok{c}\NormalTok{(}\StringTok{"Black"}\NormalTok{,}
                         \StringTok{"White"}\NormalTok{,}
                         \StringTok{"Other"}\NormalTok{),}
              \AttributeTok{exclude =} \FunctionTok{c}\NormalTok{(}\StringTok{"Not Recorded"}\NormalTok{, }\StringTok{"1"}\NormalTok{, }\StringTok{"2"}\NormalTok{, }\StringTok{"3"}\NormalTok{)))}

\NormalTok{population\_730 }\SpecialCharTok{\%\textgreater{}\%} 
    \FunctionTok{count}\NormalTok{(race)}
\DocumentationTok{\#\# \# A tibble: 4 x 2}
\DocumentationTok{\#\#   race      n}
\DocumentationTok{\#\#   \textless{}fct\textgreater{} \textless{}int\textgreater{}}
\DocumentationTok{\#\# 1 Black 46637}
\DocumentationTok{\#\# 2 White  9026}
\DocumentationTok{\#\# 3 Other  1109}
\DocumentationTok{\#\# 4 \textless{}NA\textgreater{}    742}
\CommentTok{\# check results}
\CommentTok{\# population\_730 \%\textgreater{}\% }
  \CommentTok{\# glimpse()}
\CommentTok{\# how many patients in population from 2009{-}2010?}
\NormalTok{n\_patients\_730 }\OtherTok{\textless{}{-}}\NormalTok{ population\_730 }\SpecialCharTok{\%\textgreater{}\%}
  \FunctionTok{nrow}\NormalTok{()}
\end{Highlighting}
\end{Shaded}

\textbf{Population from 81 shifts with observational data:}

\begin{Shaded}
\begin{Highlighting}[]
\CommentTok{\# load population data from the 81 observed shifts"}
\NormalTok{population\_81 }\OtherTok{\textless{}{-}} \FunctionTok{read\_rds}\NormalTok{(}\FunctionTok{paste0}\NormalTok{(rdata\_dir, }\StringTok{"/pt\_info\_final.Rdata"}\NormalTok{))}

\NormalTok{population\_81 }\OtherTok{\textless{}{-}}\NormalTok{ population\_81 }\SpecialCharTok{\%\textgreater{}\%} 
  \FunctionTok{mutate}\NormalTok{(}\AttributeTok{race =} \FunctionTok{factor}\NormalTok{(race,}
              \AttributeTok{levels =} \FunctionTok{c}\NormalTok{(}\StringTok{"Black"}\NormalTok{,}
                         \StringTok{"White"}\NormalTok{,}
                         \StringTok{"Other"}\NormalTok{),}
              \AttributeTok{exclude =} \FunctionTok{c}\NormalTok{(}\StringTok{"Not Recorded"}\NormalTok{, }\StringTok{"1"}\NormalTok{, }\StringTok{"2"}\NormalTok{, }\StringTok{"3"}\NormalTok{)))}
\CommentTok{\# how many patients in population of 81 shifts?}
\NormalTok{n\_patients\_81 }\OtherTok{\textless{}{-}}\NormalTok{ population\_81 }\SpecialCharTok{\%\textgreater{}\%}
  \FunctionTok{nrow}\NormalTok{()}
\CommentTok{\# n\_patients\_81}
\NormalTok{population\_81 }\SpecialCharTok{\%\textgreater{}\%} 
  \FunctionTok{count}\NormalTok{(race)}
\DocumentationTok{\#\# \# A tibble: 4 x 2}
\DocumentationTok{\#\#   race      n}
\DocumentationTok{\#\#   \textless{}fct\textgreater{} \textless{}int\textgreater{}}
\DocumentationTok{\#\# 1 Black  7472}
\DocumentationTok{\#\# 2 White  1424}
\DocumentationTok{\#\# 3 Other   174}
\DocumentationTok{\#\# 4 \textless{}NA\textgreater{}    113}
\CommentTok{\# pull a column of all unique shift numbers}
\NormalTok{shift\_nums }\OtherTok{\textless{}{-}}\NormalTok{ population\_81 }\SpecialCharTok{\%\textgreater{}\%} 
  \FunctionTok{distinct}\NormalTok{(shift\_num)}
\CommentTok{\# create a vector of all shift numbers from that column}
\NormalTok{shift\_nums\_vec }\OtherTok{\textless{}{-}} \FunctionTok{pull}\NormalTok{(shift\_nums, shift\_num)}
\CommentTok{\# set seed for reproducible random sampling}
\FunctionTok{set.seed}\NormalTok{(}\DecValTok{711}\NormalTok{)}
\CommentTok{\# randomly sample 20 shifts}
\NormalTok{shift\_sample\_vec }\OtherTok{\textless{}{-}} \FunctionTok{sample}\NormalTok{(shift\_nums\_vec, }\DecValTok{20}\NormalTok{)}

\NormalTok{population\_20 }\OtherTok{\textless{}{-}}\NormalTok{ population\_81 }\SpecialCharTok{\%\textgreater{}\%} 
  \FunctionTok{filter}\NormalTok{(shift\_num }\SpecialCharTok{\%in\%}\NormalTok{ shift\_sample\_vec) }
\NormalTok{n\_population\_20 }\OtherTok{\textless{}{-}}\NormalTok{ population\_20 }\SpecialCharTok{\%\textgreater{}\%} 
  \FunctionTok{nrow}\NormalTok{()}
\end{Highlighting}
\end{Shaded}

\begin{Shaded}
\begin{Highlighting}[]
\CommentTok{\# load data for patients included in the study:}
\NormalTok{pt\_sample }\OtherTok{\textless{}{-}} \FunctionTok{read\_rds}\NormalTok{(}\FunctionTok{paste0}\NormalTok{(rdata\_dir, }\StringTok{"/pts\_all.Rdata"}\NormalTok{))}
\CommentTok{\# wrangle pt\_sample to exclude unknown/not recorded/or undefined observations:}
\NormalTok{pt\_sample }\OtherTok{\textless{}{-}}\NormalTok{ pt\_sample }\SpecialCharTok{\%\textgreater{}\%} 
  \FunctionTok{mutate}\NormalTok{(}\AttributeTok{participant\_cat =} \FunctionTok{replace\_na}\NormalTok{(job, }\StringTok{"PATIENT"}\NormalTok{), }
         \AttributeTok{race =} \FunctionTok{factor}\NormalTok{(race,}
              \AttributeTok{levels =} \FunctionTok{c}\NormalTok{(}\StringTok{"Black"}\NormalTok{,}
                         \StringTok{"White"}\NormalTok{,}
                         \StringTok{"Other"}\NormalTok{),}
              \AttributeTok{exclude =} \FunctionTok{c}\NormalTok{(}\StringTok{"Not Recorded"}\NormalTok{, }\StringTok{"1"}\NormalTok{, }\StringTok{"2"}\NormalTok{, }\StringTok{"3"}\NormalTok{)),}
         \AttributeTok{arrival\_mode =} \FunctionTok{na\_if}\NormalTok{(arrival\_mode, }\StringTok{"Unknown"}\NormalTok{),}
         \AttributeTok{disposition =} \FunctionTok{na\_if}\NormalTok{(disposition, }\StringTok{"Not Recorded"}\NormalTok{),}
         \AttributeTok{.keep =} \StringTok{"unused"}\NormalTok{) }\SpecialCharTok{\%\textgreater{}\%}
  \FunctionTok{relocate}\NormalTok{(participant\_cat, }\AttributeTok{.after =}\NormalTok{ pt\_id)}
\CommentTok{\# check results:}
\CommentTok{\# pt\_sample \%\textgreater{}\% }
\CommentTok{\#   glimpse()}
\NormalTok{pt\_sample }\SpecialCharTok{\%\textgreater{}\%} 
  \CommentTok{\# filter(participant == "Yes") \%\textgreater{}\%}
  \CommentTok{\# nrow()}
  \FunctionTok{count}\NormalTok{(race)}
\DocumentationTok{\#\# \# A tibble: 4 x 2}
\DocumentationTok{\#\#   race      n}
\DocumentationTok{\#\#   \textless{}fct\textgreater{} \textless{}int\textgreater{}}
\DocumentationTok{\#\# 1 Black  3588}
\DocumentationTok{\#\# 2 White   637}
\DocumentationTok{\#\# 3 Other    64}
\DocumentationTok{\#\# 4 \textless{}NA\textgreater{}    489}
\CommentTok{\# how many patient participants?}
\NormalTok{n\_pt\_sample }\OtherTok{\textless{}{-}}\NormalTok{ pt\_sample }\SpecialCharTok{\%\textgreater{}\%} 
  \FunctionTok{nrow}\NormalTok{()}
\CommentTok{\# n\_pt\_sample}
\NormalTok{ptsample\_20 }\OtherTok{\textless{}{-}}\NormalTok{   pt\_sample }\SpecialCharTok{\%\textgreater{}\%} 
  \FunctionTok{filter}\NormalTok{(shift\_num }\SpecialCharTok{\%in\%}\NormalTok{ shift\_sample\_vec)}

\NormalTok{ptsample\_20 }\SpecialCharTok{\%\textgreater{}\%} 
  \CommentTok{\# nrow()}
  \CommentTok{\# select({-}nested\_loc) \%\textgreater{}\% }
  \CommentTok{\# filter(!is.na(participant) \& !is.na(approached) \& !is.na(participant\_final)) \%\textgreater{}\% \# NA x 443}
  \FunctionTok{select}\NormalTok{(race,}
\NormalTok{         age,}
\NormalTok{         sex,}
\NormalTok{         acuity, }
\NormalTok{         arrival\_cat,}
\NormalTok{         los\_hours,) }\SpecialCharTok{\%\textgreater{}\%} 
\NormalTok{  vtable}\SpecialCharTok{::}\FunctionTok{st}\NormalTok{()}
\end{Highlighting}
\end{Shaded}

\begin{table}

\caption{\label{tab:pt_sample}Summary Statistics}
\centering
\begin{tabular}[t]{llllllll}
\toprule
Variable & N & Mean & Std. Dev. & Min & Pctl. 25 & Pctl. 75 & Max\\
\midrule
race & 959 &  &  &  &  &  & \\
... Black & 793 & 82.7% &  &  &  &  & \\
... White & 147 & 15.3% &  &  &  &  & \\
... Other & 19 & 2% &  &  &  &  & \\
age & 971 & 48.718 & 18.919 & 0 & 33 & 62 & 97\\
\addlinespace
sex & 971 &  &  &  &  &  & \\
... Female & 557 & 57.4% &  &  &  &  & \\
... Male & 414 & 42.6% &  &  &  &  & \\
acuity & 967 &  &  &  &  &  & \\
... 1 Immediate & 8 & 0.8% &  &  &  &  & \\
\addlinespace
... 2 Emergent & 354 & 36.6% &  &  &  &  & \\
... 3 Urgent & 472 & 48.8% &  &  &  &  & \\
... 4 Stable & 130 & 13.4% &  &  &  &  & \\
... 5 Non Urgent & 3 & 0.3% &  &  &  &  & \\
arrival_cat & 964 &  &  &  &  &  & \\
\addlinespace
... EMS & 294 & 30.5% &  &  &  &  & \\
... Custody & 0 & 0% &  &  &  &  & \\
... Ambulatory & 669 & 69.4% &  &  &  &  & \\
... Unknown & 1 & 0.1% &  &  &  &  & \\
los_hours & 970 & 8.639 & 7.208 & 0.464 & 4.35 & 9.279 & 49.567\\
\bottomrule
\end{tabular}
\end{table}

\begin{Shaded}
\begin{Highlighting}[]

\CommentTok{\# how many patients are in the 20 sampled shifts?}
\NormalTok{n\_shift\_sample }\OtherTok{\textless{}{-}}\NormalTok{ ptsample\_20 }\SpecialCharTok{\%\textgreater{}\%} 
  \FunctionTok{nrow}\NormalTok{()}
\CommentTok{\# what are their races?}
\NormalTok{ptsample\_20 }\SpecialCharTok{\%\textgreater{}\%} 
  \FunctionTok{count}\NormalTok{(race) }\SpecialCharTok{\%\textgreater{}\%} 
  \FunctionTok{mutate}\NormalTok{(}\StringTok{\textasciigrave{}}\AttributeTok{\%}\StringTok{\textasciigrave{}} \OtherTok{=} \FunctionTok{round}\NormalTok{(n}\SpecialCharTok{/}\FunctionTok{sum}\NormalTok{(n)}\SpecialCharTok{*}\DecValTok{100}\NormalTok{, }\DecValTok{2}\NormalTok{))}
\DocumentationTok{\#\# \# A tibble: 4 x 3}
\DocumentationTok{\#\#   race      n   \textasciigrave{}\%\textasciigrave{}}
\DocumentationTok{\#\#   \textless{}fct\textgreater{} \textless{}int\textgreater{} \textless{}dbl\textgreater{}}
\DocumentationTok{\#\# 1 Black   793 78.3 }
\DocumentationTok{\#\# 2 White   147 14.5 }
\DocumentationTok{\#\# 3 Other    19  1.88}
\DocumentationTok{\#\# 4 \textless{}NA\textgreater{}     54  5.33}
\CommentTok{\# what was their sex?}
\NormalTok{ptsample\_20 }\SpecialCharTok{\%\textgreater{}\%} 
  \FunctionTok{count}\NormalTok{(sex) }\SpecialCharTok{\%\textgreater{}\%} 
  \FunctionTok{mutate}\NormalTok{(}\StringTok{\textasciigrave{}}\AttributeTok{\%}\StringTok{\textasciigrave{}} \OtherTok{=} \FunctionTok{round}\NormalTok{(n}\SpecialCharTok{/}\FunctionTok{sum}\NormalTok{(n)}\SpecialCharTok{*}\DecValTok{100}\NormalTok{, }\DecValTok{2}\NormalTok{))}
\DocumentationTok{\#\# \# A tibble: 3 x 3}
\DocumentationTok{\#\#   sex        n   \textasciigrave{}\%\textasciigrave{}}
\DocumentationTok{\#\#   \textless{}fct\textgreater{}  \textless{}int\textgreater{} \textless{}dbl\textgreater{}}
\DocumentationTok{\#\# 1 Female   557 55.0 }
\DocumentationTok{\#\# 2 Male     414 40.9 }
\DocumentationTok{\#\# 3 \textless{}NA\textgreater{}      42  4.15}
\CommentTok{\# what was their age?}
\NormalTok{ptsample\_20 }\SpecialCharTok{\%\textgreater{}\%} 
  \FunctionTok{filter}\NormalTok{(}\SpecialCharTok{!}\FunctionTok{is.na}\NormalTok{(age)) }\SpecialCharTok{\%\textgreater{}\%} 
  \FunctionTok{summarise}\NormalTok{(}\AttributeTok{age\_mean =} \FunctionTok{mean}\NormalTok{(age), }
            \AttributeTok{sd\_age =} \FunctionTok{sd}\NormalTok{(age),}
            \AttributeTok{age\_min =} \FunctionTok{min}\NormalTok{(age), }
            \AttributeTok{age\_max =} \FunctionTok{max}\NormalTok{(age))}
\DocumentationTok{\#\# \# A tibble: 1 x 4}
\DocumentationTok{\#\#   age\_mean sd\_age age\_min age\_max}
\DocumentationTok{\#\#      \textless{}dbl\textgreater{}  \textless{}dbl\textgreater{}   \textless{}dbl\textgreater{}   \textless{}dbl\textgreater{}}
\DocumentationTok{\#\# 1     48.7   18.9       0      97}
\end{Highlighting}
\end{Shaded}

\begin{Shaded}
\begin{Highlighting}[]
\NormalTok{staff\_all }\OtherTok{\textless{}{-}} \FunctionTok{read\_rds}\NormalTok{(}\FunctionTok{paste0}\NormalTok{(rdata\_dir, }\StringTok{"/staff\_all.Rdata"}\NormalTok{))}
\CommentTok{\# pull data for staff observed during the 20 sampled shifts}
\NormalTok{staff\_sample }\OtherTok{\textless{}{-}}\NormalTok{ staff\_all }\SpecialCharTok{\%\textgreater{}\%} 
  \FunctionTok{filter}\NormalTok{(shift\_num }\SpecialCharTok{\%in\%}\NormalTok{ shift\_nums\_vec)}
\CommentTok{\# group by sid to count individual staff}
\NormalTok{n\_staff\_sample }\OtherTok{\textless{}{-}}\NormalTok{ staff\_sample }\SpecialCharTok{\%\textgreater{}\%} 
  \FunctionTok{ungroup}\NormalTok{() }\SpecialCharTok{\%\textgreater{}\%} 
  \FunctionTok{distinct}\NormalTok{(sid) }\SpecialCharTok{\%\textgreater{}\%} 
  \FunctionTok{nrow}\NormalTok{()}

\CommentTok{\# glimpse(staff\_sample)}
\end{Highlighting}
\end{Shaded}

\hypertarget{abstract}{%
\section{Abstract}\label{abstract}}

\hypertarget{background}{%
\subsection{Background:}\label{background}}

The emergency department (ED) is a complex social setting in which
clinical care and patients' experiences depend on interpersonal
encounters, or clinical interactions. Clinical interactions are
multidimensional social phenomena embedded at the center of observable
healthcare services. Clinical interactions are also one of the primary
manifestations of healthcare disparities related to racism and
discrimination. One roadblock to understanding the role of clinical
interactions in ED disparities is a lack of methods available to study
complex networks of interacting individuals. Social network analysis of
close-proximity interactions data generated by real-time location
systems is considered a powerful tool for studying social interactions.
The purpose of this study is to describe a network analysis of clinical
interactions (NACI) and the effects of patient race on structural
network variables in an urban ED in the Southern United States.

\hypertarget{method}{%
\subsection{Method:}\label{method}}

This secondary analysis of clinical network data describes the ED
clinical interactions for patients by self-identified race and gender
groups. The Emory Institutional Review Board approved both the original
study and this analysis. Radiofrequency identification (RFID) tags worn
by consenting ED clinicians and patients were used to passively observe
participants' locations while in the ED. Location was identified every
ten seconds for two randomly selected shifts per week over one year
(July, 2009 through June, 2010). Clinical interactions were defined
where two or more participants were in the same room and in close
proximity (within 1m). Demographic and clinical data for the population
and participants were pulled from medical records. Demographic
information was not collected for ED staff other than job category (RN,
MD, and Staff) to maintain participant confidentiality. ED staff were
consented at the beginning of the study, and research staff intended to
approach all patients as possible during the preselected shifts. \#\#
Results:

104 shifts were randomly selected for observation, one day and one night
shift per week, from a total of 730 shifts (two shifts per day for one
year). 23 shifts were considered ineligible due to unforeseeable issues
like equipment failure and sick research staff. From the remaining 81
shifts,

\hypertarget{racialized-network-structures-of-clinical-interactions-in-the-ed}{%
\section{Racialized Network Structures of Clinical Interactions in the
ED}\label{racialized-network-structures-of-clinical-interactions-in-the-ed}}

\hypertarget{introduction-and-background}{%
\subsection{Introduction and
Background}\label{introduction-and-background}}

Understanding racial disparities in health and healthcare is an ongoing
priority for health services research investigators and institutions
(Agency for Healthcare Research and Quality, 2020; AMA, 2021;
Association of Academic Chairs of Emergency Medicine, 2020; Borrell \&
Vaughan, 2019; Carnethon et al., 2017; Doubeni, 2021; Hagle et al.,
2021; National Institutes of Health, 2021a, 2021b; Newman et al., 2021).
Structural, cultural, and interpersonal factors lead to racial
healthcare disparities. Racial healthcare disparities are ``differences
in the quality of healthcare that are not due to access-related factors
or clinical needs, preferences, and appropriateness of intervention''
(Institute of Medicine, 2003, p.~4). Clinical interactions (CIs) are
interpersonal encounters in the process of healthcare process with at
least two individuals. Upon presentation to the emergency department
(ED), people seeking help are embedded in a complex network of clinical
interactions involving every staff member, clinician, caregiver, and
patient present. Network science draws on the perspectives,
philosophies, and methods of various qualitative and quantitative
research traditions to understand relationships between objects embedded
in complex systems. Network analyses of clinical interactions (NACI)
have become increasingly common in the health services research
literature over the past few decades (Benton et al., 2015). Health
services researchers report correlations between hospitalizes patients'
clinical interaction network variables, cost of care, hospital
characteristics, and patient demographics including age and sex (Abbasi
et al., 2012). Focusing a network science lens on clinical interactions
in the complex emergency department (ED) environment may add to current
evidence of the social and interpersonal dimensions of racial healthcare
disparities. The purpose of this paper is to report a secondary data
analysis of clinical interactions in the Emergency Department of a large
academic hospital in the Southern U.S. using social network analyses of
ED clinical interactions between patients and healthcare personnel.

\hypertarget{finding-meaning-in-social-network-structure}{%
\subsubsection{Finding Meaning in Social Network
Structure}\label{finding-meaning-in-social-network-structure}}

The meaning of network structures depends on researchers' decisions
about how to define individual actors (i.e., nodes) and their
inter-relational connections (i.e., ties). In fact, the research
question under investigation is considered the guiding force behind
conceptual and operational interpretations of networks. Consider, for
example, data from a novel contact tracing app designed to identify
exposures. This hypothetical network would consist of all individuals
with the app who are at risk of exposure (i.e., vertices, nodes,
actors), their interactions with other nodes (i.e., ties, edges), and
exposure status (i.e., node/vertex attribute). In this example, node
nj's risk of exposure increases in relation to the number of
interactions nj has with others in the network. The relative number of
interactions, or ties, nj has compared to others in the network is a
basic measure of network centrality. As more attribute data are
available, say vaccination status, the meaning of network structures,
like centrality, must be reconsidered. When studying ED network data, we
can assume that a tie between patient and clinician nodes is a part of
service delivery: specifically, a patient-clinician interactions (i.e.,
clinical interactions). In some healthcare settings, clinical
interactions are the building blocks of clinical relationships with
broader implications for quality and equity. Clinical relationships are,
however, beyond the purview of this paper. By defining network
connections as clinician-patient interactions, its centrality is a
measure of--at the least--quantity of direct clinical care, and--we
hypothesize--important dimensions of quality clinical care like
clinician-patient communication, changes in patient acuity, and ED
throughput. Passive Location Collection with Radiofrequency
Identification (RFID) RFID systems are used in a number of hospitals
nationwide (Page, 2007), commonly for supply chain management, passive
patient identification, safe medication administration, patient
tracking, and asset tracking (Yao et al., 2012). RFID systems consist of
small tags or badges that emit radio-frequencies that are picked up by
sensors located strategically so that every RFID badge is always
identifiable to at least 3 sensors for location by triangulation (Yao et
al., 2012). The presence of RFID-enabled healthcare facilities
nationwide will allow for replication of this research, and the
evaluation of equity improvement interventions can be done with some
minor alterations to existing hardware. Compared to other technologies
used for locating resources, the inexpensive and unobtrusive design of
RFID tags make them ideal for healthcare applications. Additionally,
person-to-person proximity data, such as ours, is likely the most
informative sensor-generated data for mapping and studying human
networks (Pentland, 2012). We report a secondary analysis of
longitudinal ED contact network data. Data were collected previously
using a prospective longitudinal observational study design. The purpose
of the parent study was to describe contact characteristics among
patients and staff in the ED of a busy urban hospital to help inform
cross-infection control measures (Lowery-North et al., 2013).

\hypertarget{method-1}{%
\subsection{Method}\label{method-1}}

This secondary analysis and its parent study independently received
ethics approval by institutional review board. Sampling strategy and
data collection methods were described previously (Lowery-North et al.,
2013).

\hypertarget{sampling}{%
\subsubsection{Sampling}\label{sampling}}

Sampling strategies in network research are based on the research
question (\textbf{borgatti2013?}). Sampling strategies in network
science defer to the boundaries of the network under investigation and
the nature of the relationships between individual nodes
(\textbf{borgatti2013?}).

Data were collected with a random sampling of one day shift and one
night shift from a single academic urban ED every week over the course
of one year. A total of 104 shifts of data were collected for the
original study (Lowery-North et al., 2013). Investigators chose this
sampling strategy to minimize sampling bias related to variability in
seasonal, weekly,and daily in census, acuity, and ED staffing patterns
(Lowery-North et al., 2013).

Data points included detailed ED patient demographic and clinical
information and basic information about clinicians. Clinicians were
categorized into three professional groups; medical doctors (MD),
registered nurses (RNs), and non-clinical staff. Patient data were
collected for the following variables: race (Black, Hispanic, White, and
other), sex, age, arrival mode (ambulatory, emergency medical transport,
or in police custody), chief complaint, acuity (Emergency Severity
Index), and disposition (admitted to hospital, discharged to community,
left without being seen, or left against medical advice).

\textbf{\emph{Acuity:}} Patient Acuity was measured in the parent study
with the Emergency Severity Index (ESI) version 3. The ESI was designed
to accurately predict ED patients' resource needs and disposition
(Tanabe, Gimbel, Yarnold, \& Adams, 2004; Tanabe, Gimbel, Yarnold,
Kyriacou, et al., 2004). Each patient was assigned an ESI number on a
scale of 1-5 by a triage nurse. Lower scores indicate more urgent
patient needs, higher resource consumption, and greater likelihood of
hospital admission (Tanabe, Gimbel, Yarnold, Kyriacou, et al., 2004).
The ESI has excellent interrater reliability and consistently predicts
hospital admissions with the following rates: ESI 1 (80\%), 2 (73\%), 3
(51\%), 4 (6\%), and 5 (5\%). In another study, patients with ESI levels
1-2 had higher rates of intensive care unit admissions more often than
patients at levels 3-5 (Tanabe, Gimbel, Yarnold, Kyriacou, et al.,
2004). A limitation of the ESI is that it is not measured throughout the
patient's visit, and patient conditions change over time.

\hypertarget{data-sources}{%
\subsubsection{Data Sources}\label{data-sources}}

Original investigators collected data from three sources: patients'
electronic health records (Cerner Millennium Electronic Health Record,
Cerner Corp., Kansas City, MO), a radio-frequency identification system
(Radianse Corp., Amherst, MA), and a tag identification database.

\hypertarget{analysis}{%
\subsubsection{Analysis}\label{analysis}}

We used the R open-source statistical programming language (R Core Team,
2021) with RStudio (RStudio Team, 2022) for all statistical tests and
python programming language for network analytics and visualizations.
Empirical analysis of longitudinal network data begins with basic
descriptive properties including two-dimensional network graphs and
plots of basic network statistics. Stratification by racial/ethnic
group, rather than risk-adjustment, has been shown to generate more
informative results in recent studies of healthcare disparities (Profit
et al., 2017).

\hypertarget{results}{%
\subsection{Results}\label{results}}

Demographics for the whole ED patient population in the year in which
data were collected were reported previously (\textbf{Lowery-North?}).
In summary, the ED had 57514 patient visits from July 2009 through June
2010 and 9183 (16\%) visits during the 81 observed shifts. Of visits
that occurred during the observed shifts, 34\% (3112) were not
approached by the research team, 16\% (941) were excluded for reasons
related to the patient (e.g., refused or unable to consent), and 6\%
(389) had technical issues which precluded inclusion. Data were,
therefore, available for 4778 (52\%) patients from 81 shifts, of which
1013 (21.2\%).

\hypertarget{staff-sample}{%
\subsubsection{Staff Sample}\label{staff-sample}}

88 (84.6\%) of 104 eligible staff consented to participate, and all 88
(100\%) of those were present during at least one of the 20 shifts
sampled for this analysis (\textbf{Lowery-North?}).

\hypertarget{refs}{}
\begin{CSLReferences}{1}{0}
\leavevmode\vadjust pre{\hypertarget{ref-lowery-north2013}{}}%
Lowery-North, D. W., Hertzberg, V. S., Elon, L., Cotsonis, G., Hilton,
S. A., Vaughns 2nd, C. F., Hill, E., Shrestha, A., Jo, A., \& Adams, N.
(2013). Measuring social contacts in the emergency department {[}Journal
Article{]}. \emph{PLoS One}, \emph{8}(8), e70854.
\url{https://doi.org/10.1371/journal.pone.0070854}

\leavevmode\vadjust pre{\hypertarget{ref-profit2017}{}}%
Profit, J., Gould, J. B., Bennett, M., Goldstein, B. A., Draper, D.,
Phibbs, C. S., \& Lee, H. C. (2017). Racial/ethnic disparity in NICU
quality of care delivery {[}Journal Article{]}. \emph{Pediatrics},
\emph{140}(3). \url{https://doi.org/10.1542/peds.2017-0918}

\leavevmode\vadjust pre{\hypertarget{ref-r}{}}%
R Core Team. (2021). \emph{R: A language and environment for statistical
computing}. R Foundation for Statistical Computing.
\url{https://www.R-project.org/}

\leavevmode\vadjust pre{\hypertarget{ref-rstudio}{}}%
RStudio Team. (2022). \emph{RStudio: Integrated development environment
for r}. RStudio, PBC. \url{http://www.rstudio.com/}

\end{CSLReferences}

\end{document}
